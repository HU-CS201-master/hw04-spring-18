\documentclass[addpoints]{exam}

\usepackage{amsmath}
\usepackage{hyperref}
\usepackage{tikz}

% Header and footer.
\pagestyle{headandfoot}
\runningheadrule
\runningfootrule
\runningheader{CS 201 DS II}{Homework 4}{Spring 2018}
\runningfooter{}{Page \thepage\ of \numpages}{}
\firstpageheader{}{}{}

\boxedpoints
\printanswers

\title{Habib University\\CS 201 Data Structures II\\Spring 2018}
\author{Don't Grade Me}  % replace with your ID, e.g. sh01703
\date{Homework 4\\\numpoints\ points. Due: 19h; Monday, 12 Mar}

\begin{document}
\maketitle

\begin{questions}

  \question[10] % R-13.14 
  Draw the compact representation of the suffix trie for the string: ``minimize minime''.

  \begin{solution}
    % Write your solution here
  \end{solution}

  \question[10]% C-13.43
  Give an efficient algorithm for deleting a string from a standard trie and
  analyze its running time.

  \begin{solution}
    % Write your solution here
  \end{solution}
  
  \question[10]  % C-13.45
  Describe an algorithm for constructing the compact representation of a suffix trie, given its noncompact representation, and analyze its running time.

  \begin{solution}
    % Write your solution here
  \end{solution}

  \begin{EnvUplevel}
    The following questions refer to the 2 tables below.

    \begin{tabular}{c@{\hspace{50pt}}c}
      \begin{tabular}{|l|r|r|}
        \hline
        term & df$_t$ & idf$_t$\\\hline
        car & 18165 & 1.65\\
        auto & 6723 & 2.08\\
        insurance & 19241 & 1.62\\
        best & 25235 & 1.5\\
        \hline
      \end{tabular}
             &
               \begin{tabular}{|l|r|r|r|}
                 \hline
                 & Doc 1 & Doc 2 & Doc 3\\\hline
                 car & 27 & 4 & 24\\
                 auto & 3 & 33 & 0\\
                 insurance & 0 & 33 & 29\\
                 best & 14 & 0 & 17 \\\hline
               \end{tabular}\\
      Table 1 & Table 2
    \end{tabular}
  \end{EnvUplevel}

  \question[10]
  Consider the table of term frequencies for 3 documents denoted Doc1, Doc2, Doc3 in Table 2. Compute the tf-idf weights for the terms {\tt car, auto, insurance, best}, for each document, using the idf values from Table 1.

  \begin{solution}
    % Write your solution here
  \end{solution}

  \question[10]
  \label{q:weights}
  Compute the Euclidean normalized document vectors for each of the documents in Table 2, where each vector has four components, one for each of the four terms.

  \begin{solution}
    % Write your solution here
  \end{solution}

  \question
  With term weights as computed in the Question \ref{q:weights}, rank the three documents from Table 2 by computed score for the query {\tt car insurance}, for each of the following cases of term weighting in the query.
  \begin{parts}
    \part[5] The weight of a term is 1 if present in the query, 0 otherwise.

    \begin{solution}
      % Write your solution here
    \end{solution}
    \part[5] Euclidean normalized idf.

    \begin{solution}
      % Write your solution here
    \end{solution}
  \end{parts}

  \qformat{{\large\bf \thequestiontitle}\hfill[\totalpoints\ points]}
  \titledquestion{Programming Questions}
  The python skeleton files for each of the folloiwng will be added shortly.
  \begin{parts}
    \part Code a class to represent a generalized suffix tree. Words are added to it one at a time and it supports the usual query functions. Test it on a sample word list, e.g. the one at \url{http://thinkpython2.com/code/words.txt}.
    \part Code a class to represent an invertex index. Documents are added to it one at a time and it supports the usual query functions. Test it on a sample corpus, e.g. \href{https://archive.ics.uci.edu/ml/datasets/Reuters+RCV1+RCV2+Multilingual\%2C+Multiview+Text+Categorization+Test+collection}{Reuters RCV1}.
  \end{parts}      

\end{questions}

\end{document}